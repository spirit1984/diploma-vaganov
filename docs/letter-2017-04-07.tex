\documentclass[11pt,a4paper,oneside]{article}
\usepackage[T2A]{fontenc}
\usepackage[utf8]{inputenc}
\usepackage[english,russian]{babel}
\usepackage{expdlist}
\usepackage{graphicx}
\usepackage{amsmath}
\usepackage{amssymb}
\usepackage{algorithm}
\usepackage{algorithmic}
\usepackage{euscript}
\usepackage{amsthm}
\usepackage[pdftex,hyperindex,unicode]{hyperref}
\usepackage{cmap}
\hypersetup{
  pdftitle           = {Концепция системы},
  pdfauthor          = {Дмитрий Шевченко},
  pdfsubject         = {Ведомость},
  pdfstartview       = {FitH},
  pdfborder          = {0 0 0},
  bookmarksopen      = true,
  bookmarksnumbered  = true,
  bookmarksopenlevel = 2
}

\hoffset=-25mm \voffset=-35mm \textheight=250mm \textwidth=175mm
\sloppy
\begin{document}

\newtheorem{example}{Пример}
\newtheorem{task}{Задача}
\newtheorem{heuristic}{Задача}
\title{Концепция системы}
\author{Шевченко Д.В.}

\maketitle

Итак, вкратце, какова задача:

\begin{task}
Имеется текст $T$ а также дан образец $P$. И текст, и образец состоят из символов алфавита $\{T,G,A,C\}$. Необходимо определить расположение образца в тексте с учетом того, что в образец либо текст могли быть внесены вставки, замены и удаления символов. Необходимо определить помимо расположение образца последовательность операций минимального размера, благодаря которой этот образец можно получить из подстроки текста.
\end{task}

\begin{example}
	Пусть $T=aaaabbbb$, а $P=ab$. Тогда образец входит в текст с позиции 4, а минимальная стоимость равна нулю.
\end{example}

\begin{example}
	Пусть $T=aaaabbbb$, а $P=bab$. Тогда образец входит в текст с позиции 4, а минимальная стоимость равна 1 - одному удалению из образца символа $b$ в начале.
\end{example}

В основе идеи того, как это искать, лежит очень простая идея. Пусть $|T|=m$ (т.е. текст состоит из $m$ символов), а $|P|=n$ (обычно при картировании генома $n\ll m$). Определим функцию $f(p,t)$ как минимальную стоимость для получения подстроки образца с первого символа до $p$-го включительно, из любой подстроки текста с 1 по $t$ включительно. Заодно определим $\delta(x,y)$, как функцию стоимости замены одного символа на другой:

\begin{equation}
	\delta(x,y) = \begin{cases}
	1,\ x\neq y\\
	0,\ x = y\\
	\end{cases}
\end{equation}

Эта функция нужна для того, чтобы проще было описать функцию стоимости. Тогда функцию стоимости можно определить так:


\begin{equation}
	f(p,t) = \begin{cases}
	0,\ p = 0\\
	1+f(p-1,t),\ t = 0\\
	\min(f(p,t-1), 1+f(p-1,t), 1+f(p,t-1), f(p-1,t-1)+\delta(P_p,T_t) )\ otherwise\\
	\end{cases} 
\end{equation}

Разберем детально, что описывает эта функция. Случай первый $f(0,t)$ - это когда образец из 0 символов (т.е. пустую строку) нужно получить из подстроки $T$, очевидно, что стоимость этого равна нулю, поскольку пустая строка является по определению подстрокой любой строки. Второй случай - это когда образец нужно получить из пустой строки, очевидно, это можно сделать только удалением всех символов из образца. Несложно видеть, что это рекурсивное определение в таком случае:

\begin{equation}
	f(p,0) = p
\end{equation}

Далее, последний кейс - когда непустую подстроку образца мы ищем в непустой подстроке текста. Здесь имеются следующие варианты, из которых нужно выбрать один с минимальной стоимостью. Первый кейс - когда мы игнорируем последний символ подстроки $T$ и ищем подстроку образцу в начале подстроки. Второй кейс - когда из подстроки $P$ удаляется один символ. Третий кейс - когда из подстроки $T$ удаляется один символ. Наконец четвертый кейс - когда символ образца заменяется на символ текста - при этом, если они равны, стоимость такой замены равна нулю. Тогда, минимальная стоимость равняется:

\begin{equation}
	\min_{x\in 1\ldots m} f(n,x)
\end{equation}

Сама идея проста, однако есть некоторые проблемы. Базово, размер текста для генома вируса полноценного занимает порядка $2\cdot 10^5$ символов, а образец обычно порядка $10^4$ символов. Очевидно, что если реализовывать это напрямую, то потребуется памяти:

\begin{equation}
	O(n\cdot m) = O(2\cdot 10^5\cdot 10^4) = O(2\cdot 10^9)
\end{equation}

Т.е. около 2 гигабайт оперативной памяти. Очевидно, это много. Есть несколько эвристик, которые позволяют это резко уменьшить и будут описаны далее.

\end{document}
